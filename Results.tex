% Options for packages loaded elsewhere
\PassOptionsToPackage{unicode}{hyperref}
\PassOptionsToPackage{hyphens}{url}
%
\documentclass[
  english,
  man]{apa6}
\title{Results}
\author{Vivian Almaraz\textsuperscript{1}, AJ Haller\textsuperscript{1}, \& Alejandra Munoz\textsuperscript{1}}
\date{}

\usepackage{amsmath,amssymb}
\usepackage{lmodern}
\usepackage{iftex}
\ifPDFTeX
  \usepackage[T1]{fontenc}
  \usepackage[utf8]{inputenc}
  \usepackage{textcomp} % provide euro and other symbols
\else % if luatex or xetex
  \usepackage{unicode-math}
  \defaultfontfeatures{Scale=MatchLowercase}
  \defaultfontfeatures[\rmfamily]{Ligatures=TeX,Scale=1}
\fi
% Use upquote if available, for straight quotes in verbatim environments
\IfFileExists{upquote.sty}{\usepackage{upquote}}{}
\IfFileExists{microtype.sty}{% use microtype if available
  \usepackage[]{microtype}
  \UseMicrotypeSet[protrusion]{basicmath} % disable protrusion for tt fonts
}{}
\makeatletter
\@ifundefined{KOMAClassName}{% if non-KOMA class
  \IfFileExists{parskip.sty}{%
    \usepackage{parskip}
  }{% else
    \setlength{\parindent}{0pt}
    \setlength{\parskip}{6pt plus 2pt minus 1pt}}
}{% if KOMA class
  \KOMAoptions{parskip=half}}
\makeatother
\usepackage{xcolor}
\IfFileExists{xurl.sty}{\usepackage{xurl}}{} % add URL line breaks if available
\IfFileExists{bookmark.sty}{\usepackage{bookmark}}{\usepackage{hyperref}}
\hypersetup{
  pdftitle={Results},
  pdfauthor={Vivian Almaraz1, AJ Haller1, \& Alejandra Munoz1},
  pdflang={en-EN},
  hidelinks,
  pdfcreator={LaTeX via pandoc}}
\urlstyle{same} % disable monospaced font for URLs
\usepackage{graphicx}
\makeatletter
\def\maxwidth{\ifdim\Gin@nat@width>\linewidth\linewidth\else\Gin@nat@width\fi}
\def\maxheight{\ifdim\Gin@nat@height>\textheight\textheight\else\Gin@nat@height\fi}
\makeatother
% Scale images if necessary, so that they will not overflow the page
% margins by default, and it is still possible to overwrite the defaults
% using explicit options in \includegraphics[width, height, ...]{}
\setkeys{Gin}{width=\maxwidth,height=\maxheight,keepaspectratio}
% Set default figure placement to htbp
\makeatletter
\def\fps@figure{htbp}
\makeatother
\setlength{\emergencystretch}{3em} % prevent overfull lines
\providecommand{\tightlist}{%
  \setlength{\itemsep}{0pt}\setlength{\parskip}{0pt}}
\setcounter{secnumdepth}{-\maxdimen} % remove section numbering
% Make \paragraph and \subparagraph free-standing
\ifx\paragraph\undefined\else
  \let\oldparagraph\paragraph
  \renewcommand{\paragraph}[1]{\oldparagraph{#1}\mbox{}}
\fi
\ifx\subparagraph\undefined\else
  \let\oldsubparagraph\subparagraph
  \renewcommand{\subparagraph}[1]{\oldsubparagraph{#1}\mbox{}}
\fi
% Manuscript styling
\usepackage{upgreek}
\captionsetup{font=singlespacing,justification=justified}

% Table formatting
\usepackage{longtable}
\usepackage{lscape}
% \usepackage[counterclockwise]{rotating}   % Landscape page setup for large tables
\usepackage{multirow}		% Table styling
\usepackage{tabularx}		% Control Column width
\usepackage[flushleft]{threeparttable}	% Allows for three part tables with a specified notes section
\usepackage{threeparttablex}            % Lets threeparttable work with longtable

% Create new environments so endfloat can handle them
% \newenvironment{ltable}
%   {\begin{landscape}\centering\begin{threeparttable}}
%   {\end{threeparttable}\end{landscape}}
\newenvironment{lltable}{\begin{landscape}\centering\begin{ThreePartTable}}{\end{ThreePartTable}\end{landscape}}

% Enables adjusting longtable caption width to table width
% Solution found at http://golatex.de/longtable-mit-caption-so-breit-wie-die-tabelle-t15767.html
\makeatletter
\newcommand\LastLTentrywidth{1em}
\newlength\longtablewidth
\setlength{\longtablewidth}{1in}
\newcommand{\getlongtablewidth}{\begingroup \ifcsname LT@\roman{LT@tables}\endcsname \global\longtablewidth=0pt \renewcommand{\LT@entry}[2]{\global\advance\longtablewidth by ##2\relax\gdef\LastLTentrywidth{##2}}\@nameuse{LT@\roman{LT@tables}} \fi \endgroup}

% \setlength{\parindent}{0.5in}
% \setlength{\parskip}{0pt plus 0pt minus 0pt}

% \usepackage{etoolbox}
\makeatletter
\patchcmd{\HyOrg@maketitle}
  {\section{\normalfont\normalsize\abstractname}}
  {\section*{\normalfont\normalsize\abstractname}}
  {}{\typeout{Failed to patch abstract.}}
\patchcmd{\HyOrg@maketitle}
  {\section{\protect\normalfont{\@title}}}
  {\section*{\protect\normalfont{\@title}}}
  {}{\typeout{Failed to patch title.}}
\makeatother
\shorttitle{Results}
\DeclareDelayedFloatFlavor{ThreePartTable}{table}
\DeclareDelayedFloatFlavor{lltable}{table}
\DeclareDelayedFloatFlavor*{longtable}{table}
\makeatletter
\renewcommand{\efloat@iwrite}[1]{\immediate\expandafter\protected@write\csname efloat@post#1\endcsname{}}
\makeatother
\usepackage{csquotes}
\ifXeTeX
  % Load polyglossia as late as possible: uses bidi with RTL langages (e.g. Hebrew, Arabic)
  \usepackage{polyglossia}
  \setmainlanguage[]{english}
\else
  \usepackage[main=english]{babel}
% get rid of language-specific shorthands (see #6817):
\let\LanguageShortHands\languageshorthands
\def\languageshorthands#1{}
\fi
\ifLuaTeX
  \usepackage{selnolig}  % disable illegal ligatures
\fi


\affiliation{\vspace{0.5cm}\textsuperscript{1} Smith College}

\begin{document}
\maketitle

\hypertarget{results}{%
\section{Results}\label{results}}

\hypertarget{relationship-satisfaction-and-partner-objectifcation}{%
\subsection{Relationship Satisfaction and Partner Objectifcation}\label{relationship-satisfaction-and-partner-objectifcation}}

Figure 1 shows the direct association between predictors other objectification of a man and the woman's other objectification, and woman's relationship satisfaction and the relationship satisfaction of the man as the response variable. Figure 1 shows there were no significant associations between a man's objectification of his partner with the woman's relationship satisfaction in romantic relationships (b = 0.011, p = 0.846), and between the woman's objectification of her partner with the man's relationship satisfaction in romantic relationship (b = 0.002, p = 0.967) (see Figure 1.). Hypothesis 1 is not supported as the results show there is not a statistically significant relationship between other objectification and relationship satisfaction for both men and women. Hypothesis 2 is not supported as the results show there are not statistically significant differences in other objectification and relationship satisfaction between gender. Please refer to table 1 for supplemental information.

\hypertarget{relationship-satisfaction-with-mediation}{%
\subsection{Relationship Satisfaction with Mediation}\label{relationship-satisfaction-with-mediation}}

Models 1 and 2 includes the variable self objectification for both the man and the woman as a mediator between the other objectification of a man with the woman's relationship satisfaction (model 1), and the associations between other objectification of a woman with the man's relationship satisfaction (model 2).

In Model 1, there is not a significant association between the man's partner objectification towards the woman with the woman's relationship satisfaction while mediating for the woman's objectification of herself (b = 0.039, p = 0.519)(see Figure 1.). This does not support our third hypothesis which suggested that the association between man's objectification of his partner and the partner's relationship satisfaction is dependent on a woman's objectification of herself. However, model 1 did result in a statistically significant association between the man's partner objectification and the woman's self objectification (b = 0.279, p = 0.0002)(see Figure 1.). This suggests that a man who objectifies the woman in a heterosexual relationship is associated with the woman objectifying herself more.

Model 2 shows that there is no significant association between a woman's objectification of her partner and the partner's relationship satisfaction with the man's self objectification as a mediating variable (b = 0.026, p = 0.687)( see Figure 1.). This does not support our fourth hypothesis that the association between a woman's objectification of her partner and the partner's relationship satisfaction is dependent on a man's objectification of himself. However, Model 2 also shows that there is a statistically significant indirect relationship between the woman's partner objectification and the man's self objectification (b = 0.209, p = 0.011). This suggests that a woman's partner objectification towards the man makes men in heterosexual relationships objectify themselves more. Results in model 2 also show that a man's objectification of himself is associated with a decrease in his relationship satisfaction (b = -0.150, p = 0.015)(see Figure 1.).

\begin{figure}
\centering
\includegraphics{Images/hypothesis_model_results.png}
\caption{The Moderated Mediation Model: This figure shows the associations between predictors, mediators, and response variables as the moderated mediation model. Model 1 is represented by green arrows, and model 2 is represented by blue arrows. Dashed arrows are statistically insignificant coefficients (B) demonstrating p values \textgreater{} 0.05. Bold arrows and asterix are statistically significant coefficients (B) with p values \textless{} 0.05.}
\end{figure}

\begin{figure}
\centering
\includegraphics{Images/SO_vs_RSA.png}
\caption{Self Objectification Vs. Relationship Satisfaction}
\end{figure}


\end{document}
